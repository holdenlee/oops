
\section{Critical and near-critical percolation (Gady Kozma)}

Critical and near-critical percolation is well-understood in dimension 2 and in high dimensions. The behaviour in intermediate dimensions (in particular 3) is still largely not understood, but in recent years there was some progress in this field, with contributions by van den Berg, Cerf, Duminil-Copin, Tassion and others. We will survey this recent progress (and a few older but not sufficiently known results).

Prerequisites: The Fortuin-Kasteleyn-Ginibre (FKG) and van den Berg-Kesten (BK) inequalities. See e.g. Chapter 2 of Duminil-Copin's lecture notes \url{https://www.ihes.fr/~duminil/publi/2017percolation.pdf} or almost any book on percolation theory. 

Slides: \url{https://www.math.ubc.ca/Links/OOPS/slides/Kozma_1.pdf}

\subsection*{2020/5/25 Lecture 1}

Let's start with basic definitions. Examine the graph $\Z^d$ for $d\ge 2$, with edges between nearest neighbors (in $L^1$ norm). I'm most interested in dimension 3.

For $p\in [0,1]$, keep every edge with probability $p$ and delete it withprobability $1-p$, independently for each edge. For background, see Grimmett's book. (The book is a little outdated; I'll discuss results not in the book.) %none of the results I prove will be in the book.)

There is some $p_c\in (0,1)$ (critical $p$) such that for $p<p_c$, all components (``clusters") of the resulting graph are finite, while for $p>p_c$, there is a unique infinite cluster. 
Existence of $p_c$ is a consequence of the 0-1 law.

This course is about what happens at $p_c$. The behaviour at and near $p_c$ is not well understood, except if $d=2$ or $d>6$. We're interested in the statistics of finite clusters, which has very peculiar properties in all cases we know.

We'll focus on recent advances. The Aizenman-Kesten-Newman argument from the 80's, originally used to prove uniqueness, plays an important role. The argument is in a paper of Cerf in 2015. This is an important piece to understand what is happening at criticality.

What's charming about percolation theory, is that you have to say what is happening at $p_c$ without knowing what the value of $p_c$ is.
From a combinatorial point of view: you have some quantity which has an exponential and polynomial correction; you want to understand the polynomial correction without knowing the exponential part. This seems impossible.

\begin{thm}
$\E_{p_c}(|\mathscr C(0)|)=\iy$.
\end{thm}
Notation: $0$ is the point with all coordinates 0, $\mathscr C(x)$ is the cluster containing $x$. The expected size of the cluster is infinity. Even if it is always finite, it has large tails---it has no finite first moment.
\begin{proof}
Fix $p$ and denote $\chi = \E_p(|\mathscr C(0)|)$. Let
\begin{align*}
\ep&< \rc{4d\chi}
\end{align*}•
We will show that at $p+\ep$ there is no infinite cluster. Consider $p+\ep $ percolation as if we (1) take $p$-percolation and (2) then ``sprinkle" each edge with probability $\ep$. For a vertex $x$ and a sequence of directed edges $e_1,\ldots ,e_n$,
denote by $E_{x,e1,...,en}$ the event that 0 is connected to $x$ by a path $\gamma_1$ in $p$-percolation from 0 to $e_1$ then $e_1$ is sprinkled, then there is a path $\ga_2$ from $e_1$ to $e_2$ then $e_2$  is sprinkled and soon. We end with a path $\ga_{n+1}$ from $e_n$ to $x$. We require all the $\gamma_i$ to be \emph{disjoint}. Clearly $0\lra x$ in $p+\ep$-percolation if and only if there exist some $e_1,\ldots,e_n$ (possibly empty) such that $E_{x,e_1,\ldots, e_n}$ hold.
Then $(0\lra x) = \bigcup_{n=0}^{\iy}\bigcup_{e_1,\ldots, e_n} E_{x,e_1,\ldots, e_n}$ and by the union bound
\begin{align*}
\Pj_{p+\ep} (0\lra x) &\le \sumz n\iy \sum_{e_1,\ldots, e_n}\Pj(E_{x,e_1,\ldots, e_n}).
\end{align*}•
This is the probability in the setting where edge has 3 states: $p$, $\ep$, or closed (missing).

By the BK inequality, this is
\begin{align*}
&\le \sumz n\iy \sum_{e_1,\ldots, e_n} \Pj_p(0\lra e_1^-)\Pj_p(e_1^+\lra e_2^-)\cdots \Pj_p(e_n+\lra x)\ep^n.
\end{align*}•
(The paths are disjoint witnesses.)

Summing over all $x$ gives
\begin{align*}
\chi(p+\ep) &\le \sumz n\iy \ep^n \sum_{e_1,\ldots, e_n} \Pj_p(0\lra e_1^-)\Pj_p(e_1^+\lra e_2^-)\cdots \Pj(e_n^+\lra x).
\end{align*}•
(The number of $x$ that $e_n^+$ is connected to is the size of $|\mathscr (0)|$.)
%this is all the geometry, then straightforward args
Summing over $x$ gives one $\chi(p)$ which we can take out of the sum
\begin{align*}
\sumz n\iy \ep^n \chi(p) \sum_{e_1,\ldots, e_n} \Pj_p(0\lra e_1^-)\Pj_p(e_1^+\lra e_2^-)\cdots \Pj(e_{n-1}+\lra x).
\end{align*}•
Even though $e_n^+$ does not appear in the summand we are summing over it. $e_n^+$ has $2d$ possibilities. Summing over $e_n^-$ gives another $\chi$ term. Taking both out of the sum gives
\begin{align*}
&= \sumz n\iy \ep^n \cdot 2d\chi(p)^2 \sum_{e_1,\ldots, e_{n-1}} \Pj_p(0\lra e_1^-)\cdots \Pj_p (e_{n-2}^+\lra e_{n-1}^-)\\
&= \sumz n\iy \ep^n (2d)^n \chi(p)^{n+1}<\iy.
\end{align*}•
This shows $p+\ep_c$, contradiction. 

(Set of $p$'s for which it is finite is an open set, so at the boundary of the set $\chi$ cannot be finite. We used finiteness at the end of the proof.)

The argument also gives 
\begin{align*}
\chi(p) &\ge \rc{4d(p_c-p)}
\end{align*}•
for all $p<p_c$. 
\end{proof}
%Grimmett, p. 265 equivalent but phrased using differential inequalities instead of this more explicit geometric series argument
%Grimmett Theorem 6.1 is basically this argument. 
%This looks more robust than comparing with branching process.
%generalize to transitive graph
%transitivity, sum does not depend on $e_n^+$. Percolation on groups, take home.
%even without transitivity it gives that $\sup_v E|C(v)|$ is infinite at $p_c$.

\grbox{
van den Berg-Kesten inequality: 
%Suppose we have $n$ independent bits. 
%witness is the path itself
Let $E$ be an event on $\{\pm 1\}^n$. We say that an $A\subeq \{1,\ldots, n\}$ is a witness for $E$ (this is an event itself, denote by $\om$ the parameter) if any $\om'$ such that $\om'_i=\om_i$ for all $i\in A$ satisfies that $\om'\in E$. For example, for the event $0\to x$, any path between 0 and $x$ can serve as a witness.

For two events $E$ and $F$ we denote by $E\circ F$ the event that they have disjoint witnesses. Then 
\begin{align*}
\Pj(E\circ F)&\le \Pj(E)\Pj(F).
\end{align*}•
(van den Berg-Kesten 1985 for increasing $E$ and $F$, Reimer 1997 for arbitrary $E$ and $F$)
}

We saw we can argue at $p_c$ using a argument by contradiction. We'll see another example of this argument.

For a set $S\sub \Z^d$ denote by $\pl S$ the set of $x\in S$ with a neighbour $y\nin S$. 
\begin{thm}
Let $S\sub \Z^d$ be some finite set containing 0. Then
\begin{align*}
\sum_{x\in \pl S} \Pj_{p_c} (0\xlra S x) &\ge 1.
\end{align*}•
\end{thm}
%edges retained open, removed closed.
\begin{proof}[Proof sketch.]
Let $x\in \Z^d$. If $0\lra x$ then there exists $0=y_1,\ldots, y_n=x$ such and open paths $\ga_i$ such that 
\begin{enumerate}
\item
$\ga_i$ is from $y_i$ to $y_{i+1}$ and is contained in $y_i+S$.
\item 
The $\ga_i$ are disjoint. 
\end{enumerate}•
We have $n\ge r|x|$ for some number $r>0$ that depends on $S$. (Here $|x|$ is $L^1$ norm.)
A calculation similar to the previous proof shows that
\begin{align*}
\Pj(0\lra x) &\le \sum_{n\ge r|x|} \pa{\sum_{y\in \pl S} \Pj_{p_c}(0\xlra S y)}^n.
\end{align*}•
If the value in the parenthesis is smaller than 1, then $\Pj(0\lra x)$ decays exponentially in $|x|$, contradicting the previous theorem.
%$n^d$ vertices. finite.
\end{proof}
%first time path left ellipse. Then around that point. First time leave
%don't need finite susceptibility, exp growth

A full proof can be found in H. Duminil-Copin and V. Tassion \cite{duminil2016new}.
%A new proof of the sharpness of the phase transition for Bernoulli percolation on

\begin{thm}[Menshikov, Aizenman-Barsky]
For any $p<p_c$, $\chi(p)<\iy$. 
\end{thm}
This is a deeper theorem.

We need to know at the first step that there's one phase transition. Otherwise, there could be $p_1,p_2$, such that below $p_1$ all clusters are finite and expectations are finite, between $p_1$ and $p_2$ there is all clusters are finite but expectations are infinite, and $>p_2$, there is an infinite cluster with positive probability. That would be a completely different picture!
%identification of these critical points is cornerstone of understanding percolation.
%lemma of cerf uses this theorem

%stuff known. more corollaries. by cerf. 
%Lemma of Cerf, $\Pj_{p_c} (x\xlra{\La_{2n}}y)\ge cn^{-C}$. 
%then applications. $\Pj_{p_c} (x\stackrel{\lra}{\lra}y)\le Cn^{-C}$. Connected in 2 disjoint clusters, probability quite small. 

What happens for $d>6$? %60 page paper
Remote paths of clusters develop independently, as if the other doesn't exist. There's a lot of space. They never get to meet. 
%Results in high dimensions Cachy-Harr and Gordan-Slate

%finite range interactions, treat 7-D systems

\subsection*{2020/5/27 Lecture 2: The Aizenman-Kesten-Newman argument}

%\begin{thm}
%Let $S\sub \Z^d$ be some finite set containing 0. Then $\sum_{x\in \pl S} \Pj_{p_c} (0\xlra S x)\ge 1$. 
%\end{thm}
%Two applications:
%\begin{lem}[K-Nachmias, 2011]
%For any $x\in \pl \La_n$, $\La_n:=[-n,n]^d$,
%\begin{align*}
%\Pj_{p_c}
%\end{align*}•
%\end{lem}
%\begin{lem}[Cerf, 2015]
%
%\end{lem}
Today we'll be getting out of the '80's (though this argument is from the 80's). 

We start with the idea of using exploration. (This is the one most important takeaway from today's lecture.) We demonstrate this idea with a simple lemma. Recall $\mathscr C(0)$ is the cluster containing 0 (everything connected to 0 via open edges).
\begin{lem}
Let $E$ be the number of open edges in $\sC(0)$ and let $B$ be the number of closed edges in the boundary. There are $C$ and $c$ such that for any $n$,
\begin{align*}
\Pj_p (B+E\le n, |(1-p)E-pB|> \la \sqrt n) &\le Ce^{-c\la^2}.
\end{align*}•
\end{lem}
The important part is the $(1-p)E-pB$ part. Consider $p=\rc2$, so we have roughly the same amount of open and closed edges.
\begin{proof}
We define sets of edges $\phi = S_0\sub S_1\sub \cdots$ for $i\le n$ as follows. Assume at step $i$ there exists some edge $e\nin S_i$ such that there is an open path in $S_i$ from 0 to one of the vertices of $e$.
We choose one such $e$ arbitrarily (e.g., by lexicographic order) and define $S_{i+1}:=S_i\cup \{e\}$. (Expose the cluster one edge at a time.) If no such $e$ exists (and this happens when $|S_i|=B+E$), let $S_{i+1}=S_i$.

Let 
$$X_i = (1-p)\pat{\#open edges in $S_i$} - p\pat{\#closed edges in $S_i$}.$$
The key point is \emph{$X_i$ is a martingale} ($\E[X_{i+1}|X_i]=0$). 
%at every step no info

The lemma follows from Azuma-Hoeffding.
\end{proof}
\begin{thm}[Azuma-Hoeffding]
If $X_i$ is a martingale and if $|X_{i+1}-X_i|\le m$ for some numbers $m_i$, then for any $M$,
\begin{align*}
\Pj(|X_n-X_0|) &\le 2\exp\pf{-M^2}{2\sumo iN m_i^2}.
\end{align*}
\end{thm}

This is a flexible argument. You can start from a set of vertices (not just one), or you can add additional stopping conditions.
%complicated stopping conditions
I'll show one variation.
\begin{lem}
Let $S\sub \La$ be the set of vertices connected to the boundary. Let $E$ be the number of open edges between vertices of $S$ and let $B$ be the number of closed edges with at least one vertex in $S$ and both vertices in $\La$. Let $X=(1-p)E-pB$. Then
\begin{align*}
\Pj(|X|>\la n^{d/2}) &\le Ce^{-c\la^2}.
\end{align*}•
\end{lem}
Start from the boundary and examine edges inside the box in arbitrarily order until I discover all vertices connected to the boundary by open paths.

%Cerf 2015
\begin{df}
Let $A,B$ be subsets of $E\subeq \Z^d$. We denote by
\begin{align*}
A\stackrel{\xlra E}{\lra} B
\end{align*}•
the event that there are 2 disjoint clusters in $E$ which intersect both $A$ and $B$. We write $A\stackrel \lra\lra$ to mean $A\stackrel{\xlra E}{\lra} \pl E$.
%We write...
\end{df}
This is not the same as having 2 disjoint paths. It's stronger. Not only are they disjoint, they are in disjoint clusters. 
\begin{thm}
Let $V$ be the number of edges $(x,y)$ in $\La_n$ such that $\{x,y\}\stackrel \lra\lra \pl \La_n$, i.e., both $x$ and $y$ are connected to $\pl \La_n$ but $x\stackrel{\La_n}{\not\lra} y$. Then $\E(V)< Cn^{d-\rc2} \sqrt{\log n}$. 
\end{thm}
Clusters can meet outside $\La_n$. This proof is a simplification due to Gandolfi-Grimmett-Russo \cite{gandolfi1988uniqueness} (a nice 4-page paper).
\begin{proof}
For $S\subeq \La_n$ define $X(S)$ to be 
\begin{align*}
&(1-p)\pat{\# open edges between 2 vertices of $S$}\\
&-p\pat{\# closed edges with at least 1 vertex in $S$, both in $\La_n$}
\end{align*}
Let $\sC_1,\sC_2,\ldots $ be all the clusters in $\La_n$ that touch the boundary. Then
\begin{align*}
X\pa{\bigcup_i \sC_i} - \sum_i X(\sC_i) &=pV.
\end{align*}•
This formula is the essence of the proof.
%Let's see what an open edge contributes. 
\begin{itemize}
\item
For an open edge: It's necessarily in one cluster. It contributes $1-p$ in both terms, which cancel out.
\item
For a closed edge with both points in the same cluster: It contributes $p$ to both terms.
\item
What's left is the event that the closed edge goes between 2 clusters: It is counted twice in the sum, so contributes $-p+2p=p$.
\end{itemize}•

The exploration argument shows that with high probability
\begin{align*}
\ab{X\pa{\bigcup \sC_i}}&<Cn^{d/2}\sqrt{\log n}&
|X(\sC_i)|&<C\sqrt{|\sC_i|} \sqrt{\log n}
\end{align*}•
for all $i$.
%probabilist cheap way out
%prob poly in n

By Cauchy-Schwarz,
\begin{align*}
\sum_i \sqrt{|\sC_i|} &\le \sqrt{\sum_i |\sC_i|}\sqrt{\sum_i 1}\le \sqrt{n^d}\sqrt{n^{d-1}}=n^{d-\rc 2}.
\end{align*}•
(The number of clusters is bounded by the size of the boundary.)
With high probability can be made to mean ``with probability $>1-n^{-1/2}$" and we are done.
\end{proof}

You can get an estimate for a single edge, by using the theorem for $2n$.
\begin{cor}
For $x$ a neighbour of 0, 
\begin{align*}
\Pj(\{0,x\}\stackrel \lra\lra\pl\La_n) &< C\sfc{\log n}{n}.
\end{align*}•
\end{cor}
%one I like for no obvious reason
%Take a box
By changing from where you explore you
can get all kinds of results. For example, if L is the union of all
clusters reaching the left side of $\La_n$ and R is the union of all
clusters reaching the right side of $\La_n$ then
\begin{align*}
X(L\cup R) - X(L)-X(R)
\end{align*}•
teaches something about edges connected to both the left and
the right. Hutchcroft has a version where one explores from
random points.

\begin{thm}
Let $S\sub \Z^d$ be some finite set containing 0. Then $\sum_{x\in \pl S} \Pj_{p_c} (0\xlra S x)\ge 1$. 
\end{thm}
Two applications:
\begin{lem}[K-Nachmias, 2011]
For any $x\in \pl \La_n$, $\La_n:=[-n,n]^d$,
\begin{align*}
\Pj_{p_c}(0\xlra{\La_n} x) &\ge c\exp(-C\log^2 n).
\end{align*}•
\end{lem}
\begin{lem}[Cerf, 2015]
For any $x,y\in \La_n$, 
\begin{align*}
\Pj_{p_c} (x\xlra{\La_{2n}}y) &\ge cn^{-C}
\end{align*}•
We can take $C=2d^2-2d$.
\end{lem}
All constants $c,C$ might depend on the dimension.
\begin{proof}
Assume first that $x,y$ are on the same line (and the distance is even): $x-y=(2k,0,\ldots, 0)$, $k\le n$. By the theorem there exists $z\in \pl \La_k$ such that
\begin{align*}
\Pj(0\xlra{\La_k}z) &\ge \rc{2d|\pl \La_k|}\ge \fc{c}{k^{d-1}}.
\end{align*}•
%most interested in reflection
By rotation and reflection symmetry we may assume $z$ is in
some face of $\La_k$, for example $z_1 = k$. Let $\ol z$ be the reflection of $z$
in the first coordinate i.e. $z =(-z_1, z_2, \ldots, z_d)$. By reflection
symmetry we also have $\Pj(0\xlra{\La_k} \ol z) \ge ck^{1-d}$. Translating z to x
and $\ol z$ to $y$ gives
\begin{align*}
\Pj(x\xlra{x+\La_k} x+z) , \Pj(y\xlra{y+\La_k} y+\ol z) &\ge \fc{c}{k^{d-1}}.
\end{align*}•
But $x+z=y+\ol z$. 

Since $x+\La_k\sub \La_{2n}$ and ditto for $y+\La_k$, we can write
\begin{align*}
\Pj(x\xlra{\La_{2n}} x+z) , \Pj(y\xlra{\La_{2n}} y+\ol z) &\ge \fc{c}{k^{d-1}}.
\end{align*}•
By FKG, 
\begin{align*}
\Pj(x\xlra{\La_{2n}}y) \ge \Pj(x\xlra{\La_{2n}} x+z) , \Pj(y\xlra{\La_{2n}} y+\ol z) &\ge \fc{c}{k^{2d-2}}.
\end{align*}•

With a slightly smaller $c$, we
can remove the requirement that the distance between $x$ and $y$
is even. If they are not on a line, we define
\begin{align*}
x=x_0,\ldots, x_d=y
\end{align*}•
such that each couple $x_i,x_{i+1}$ differ by only one coordinate. Hence $\Pj(x_i\xlra{\La_{2n}}x_{i+1}) \ge cn^{2-2d}$. Using FKG again gives
\begin{align*}
\Pj(x\xlra{\La_{2n}}y) &\ge \Pj(x_0\xlra{\La_{2n}}x_1, \cdots ,x_{d-1}\xlra{\La_{2n}}x_d)\\
&\ge \prodo id \Pj(x_{i-1}\xlra{\La_{2n}} x_i) \ge \fc{c}{n^{2d^2-2d}}.
\end{align*}•
\end{proof}
\begin{thm}[Fortuin-Kasteleyn-Ginibre, Harris]
A function $f:\{\pm1\}^n$ is called increasing if it (weakly) increases in every coordinate. For any 2 increasing functions $f,g$, 
\begin{align*}
\E[fg] &\ge \E[f]\E[g].
\end{align*}•
\end{thm}


This was recently improved to $cn^{-d^2}$ by van den Berg and Don.
Their proof has an interesting topological component (Brouwer's fixed point theorem).


\begin{thm}[Cerf, 2015]
$\Pj_{p_c}(\La_{n^c} \stackrel \lra\lra \pl \La_n) \le Cn^{-c}$ for $c>0$ small enough.
More precisely, 
\begin{align*}
\Pj_{p_c}(\La_{n^{1/(8d^2+8d)-o(1)}} \stackrel \lra\lra \pl \La_n) \le Cn^{-1/4}
\end{align*}•
\end{thm}
Prove using the lemma.
%generalize for arb p
\begin{itemize}
\item
The theorem actually holds for all $p$.
%patching, analysis argument
\item
Cerf had a scheme for improving the exponents: repeat the following.
\begin{enumerate}
\item
Get a better estimate for the number of clusters from $\pl \La_{2n}$ to $\pl \La_n$.
\item
Looking at $\sum \sqrt{|\sC|}$, get a better estimate for $\Pj(\{0,x\}\stackrel \lra\lra \pl \La)$. 
\item
Get a better estimate for $\Pj(\La_{n^c} \stackrel \lra\lra \pl \La_n)$. (Use a patching argument.)
\end{enumerate}•
Unfortunately, the end result was not a big improvement over $\rc 2$.
(The exponent is $-\fc{2d^2+3d-3}{4d^2+5d-5}$.)
\end{itemize}•

%Fix $p$ and let $n\to \iy$, then known already. Not the same as saying fix $n$, and this holds uniformly for $p$.
%proof not significantly different. 

%any improvement on the 2-arm prob usually sheds some new light on 3d percolation so any improvement is interesting.
%Kesten: if get to n^{-2} can prove something spectacular?

\subsection*{2020/5/28 Lecture 3}

%\begin{lem}
%Call a cluster $\sC$ in $\La_n$ ``large" if it intersects $\fc 78$ of the cubes of side-length $n^\eta$ in $\La_n$. Then
%\begin{align*}
%\Pj_{p_c} (\exists \text{ large cluster})&\le 1-c.
%\end{align*}•
%\end{lem}
%(The exact number is not important; any number in $(0,1)$ will do.)
%This is a critical phenomenon. 
%
%\begin{thm}[Duminil-Copin-K-Tassion, unpublished]
%For $d\ge 3$ and some $\nu = \nu(d)>0$, $\Pj_{p_c}(\La_{n^\nu} \not\lra \pl \La_n) > Cn^{-d}$.
%\end{thm}
%\begin{proof}
%Examine one $\nu$ (whose value will be chosen later) and assume by contradiction that this probability is, in fact, larger than $1-Cn^{-d}$. Then, with probability $>1-n^{-d\nu}$, 
%%sum over all boxes, still small.
%\emph{each} box $a+\La_{n^\nu}$, $a\in \La_n$ is connected to $a+\pl\La_n$. Denote this event by $A$. In particular, all boxes in $\La_{n/4}$ are connected to $\pl \La_{n/2}$. 
%(During this proof, whenever we say ``cluster" we mean a cluster in $\La_n$ that intersects $\La_{n/4}$ and $\pl \La_{n/2}$. 
%%take union bound
%
%%The event $A$ implies all $n^\nu$ boxes in $\La_{n/2}$
%For every cluster $\sC$ let $N(\sC)$ be the number of $n^\nu$-subboxes of $\La_{n/2}$ that intersect $\sC$. Under $A$ we have, by concavity %Bernoulli's inequality.
%$$n^{(1-\nu)d}\lesssim \sum_{\sC} N(\sC)\le 
%\pa{\sum_{\sC} N(\sC)^{(d-1)/d}}^{d/(d-1)}.
%$$
%%how many clusters there are, what is the distribution of the size?
%I add the condition that the clusters is small.
%By the lemma (there is large probability there is no large cluster), 
%\[
%\E \sum_{\sC\text{ small}} N(\sC)^{(d-1)/d} \ge cn^{(1-\nu)(d-1)}. 
%\]
%%happen with prob close to 1.
%%0.01 there is no large
%
%%union bd factor $n^d/n^\nu$?
%%$d(1-\nu)$
%The proof of Gandolfi-Grimmett-Russo shows that
%\begin{align*}
%\Pj(0\stackrel \lra\lra \pl \La_{n/4}) &\le Cn^{-d/2} + C(\log n) n^{-d} \E \sum_{\sC}\sqrt{|\sC|}
%\end{align*}•
%%sqroot crucial
%%intermediate claim
%where the sum is over clusters in $\La_{n/2}$.
%There are technicalities with $n/4$ and $n/2$. A variation of the argument, also due to Cerf, shows that one can take the sum only over $\sC$ in $\La_n$ that intersects $\La_{n/4}$ and $\pl \La_{n/2}$. With Cerf's lemma, 
%%from 2 neighboring points
%\begin{align*}
%\Pj(\La_{2n^\nu}\stackrel \lra\lra \pl \La_{n/4}) &\le Cn^{C\nu}\Pj(0\stackrel \lra\lra \pl \La_{n/4})\\
%&\le Cn^{-d/2+C\nu} + Cn^{-d+C\nu} \E \sum_{\sC} \sqrt{N(\sC)}.
%\end{align*}•
%%second term important
%
%The isoperimetric inequality in $\Z^d$ (a version for boxes) shows that for every small $\sC$ we have at least $cN(\sC)^{(d-1)/d}$ subboxes of $\La_{n/2}$ which intersect $\sC$ but have a neighboring box that does not intersect $\sC$. (This is some kind of local isoperimetric inequality. It's important that we say ``for every small $\sC$": small boxes do not reach more than a constant fraction of boxes. If a cluster reaches all boxes, the set is empty.)
%Let $Q$ be such a box and let $Q'$ be its neighbour that does not intersect $\sC$. Under event $A$, this implies $Q\cup Q'$ is connected to distance $n/4$ by 2 disjoint clusters.
%
%There is some over-counting in this argument, every $2n^\nu$ box might be counted for every cluster that intersects it. We bound the over-counting crudely by the volume of the box, $Cn^{d\nu}$. Overall we get, under $A$,
%\begin{align*}
%\#\{\text{such boxes}\} &\ge cn^{-d\nu} \sum_{\sC\text{ small}} N(\sC)^{(d-1)/d}.
%\end{align*}•
%Taking expectations we will get a contradiction:
%\begin{align*}
%C n^{(1-\nu)d} \Pj(\La_{2n^\nu} \stackrel \lra\lra \pl \La_{n/4}) &\ge cn^{-d\nu} \E \sum_{\sC\text{ small}}N(\sC)^{(d-1)/d}/
%\end{align*}•
%Together these give 
%$$\E\sum_{\sC\text{ small}} N(\sC)^{(d-1)/d} \le Cn^{d/2+C\nu} + Cn^{C\nu} \sum_{\sC}\E\sqrt{N(\sC)}.$$
%...
%THus
%\begin{align*}
%\sum_{\sC\text{ small}} \sqrt{N(\sC)}\ge ...
%\end{align*}•
%\end{proof}
%The proof in a nutshell:
%\begin{itemize}
%\item
%The AKNC argument gives 
%\begin{align*}
%\Pj(\La_{n^\nu} \stackrel \lra\lra \La_n) &\le \text{uninteresting terms }n^{-d}\sum\sqrt{|\sC|}.
%\end{align*}•
%\item
%The contradictory assumption (all boxes are connected---sum of boundary covers everything), the isoperimetric inequality and the fact that there are no large clusters give
%\begin{align*}
%\Pj(\La_{n^\nu} \stackrel \lra\lra \La_n) &\ge \text{uninteresting terms }n^{-d}
%\sum |\sC|^{(d-1)/d}.
%\end{align*}•
%\end{itemize}•
%These are contradictory. We need $d\ge3$ to get a contradiction. In 3-D, the boundary is greater than $\sqrt{}$. %The result works in 2-D using a different argument.
%
%%1/(2d) in higher dim
%Remarks:
%\begin{itemize}
%\item
%Going through the calcuation gives $\nu < \fc{d-2}{d^3+4d^2+d-2}$ so say, $1/64$ at $d=3$.
%\item
%The theorem holds also at $d=2$ (known since the 80's, with a different proof).
%\end{itemize}•
%
%See: dependencies diagram II in slides. (Cerf's inequality is in wrong direction.) We use criticality explicitly in the bolded boxes.
%\begin{enumerate}
%\item
%Increase $p$ slightly, there is still no infinite cluster, contradict you are at criticality.
%\item
%Increase $p$ slightly, construct infinite cluster.
%\end{enumerate}•
%
%Other results:
%%p. 45
%\begin{lem}
%If there is an infinite cluster with positive probability ($\te:=\Pj(0\lra \iy)>0$) then for every $\ep>0$ there exists $n$ such that for any set $A\subeq \La_n$ intersecting both $\{0\}$ and $\pl \La_n$ we have $\Pj(A\lra \iy)>1-\ep$. ($A\lra \iy$ means $|\sC(A)|=\iy$.)
%\end{lem}
%Classical results: 
%\begin{itemize}
%\item
%Close but not quite correct statement: $n\times n\times n$ box, the probability there is a path that goes from the left to right is at least a constant. At criticality, this works in all dimensions.
%(Kesten's book on Percolation)
%%not as pleasant as Grimmett.
%\item
%At criticality, the probability of being connected to the boundary is
%$\Pj_{p_c}(0\lra \pl \La_n) > cn^{(1-d)/2}$.
%%
%\end{itemize}•
