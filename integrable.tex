
\section{Gibbsian line ensembles in integrable probability (Ivan Corwin)}

Many important models in integrable probability (e.g. the KPZ equation, solvable directed polymers, ASEP, stochastic six vertex model) can be embedded into Gibbsian line ensembles. This hidden probabilistic structure provides new tools to control the behavior and asymptotics of these systems. In my first talk, I will discuss the Airy line ensemble and its origins and properties. In my second talk, I will discuss the KPZ line ensemble and explain how this structure is used to probe the temporal correlation structure of the KPZ equation. In my final talk, I will zoom out and discuss the origins of this hidden structure. 

\begin{itemize}
\item
Webpage: \url{https://www.math.ubc.ca/Links/OOPS/abs_Corwin.php}
\item
Slides: \url{http://www.math.ubc.ca/Links/OOPS/slides/Corwin_1.png}
\item
Problems: \url{http://www.math.ubc.ca/Links/OOPS/slides/Corwin_problems.pdf}
\end{itemize}•

\subsection*{2020/6/8 Lecture 1}

Message: Gibbsian line ensembles are
\begin{itemize}
\item
interesting: come up in many probabilistic/combinatorial models and having (nontrivial) universal scaling limits.
\item
useful: tool in establishing regularity and characterizing limit behavior. %One of the themes is that if you understand marginal information, and have a gaugue property, then you get spatial and temporal regularity with a bit of work.
\end{itemize}•

Talks:
\begin{enumerate}
\item
In this talk I'll focus on non-touching geometric random walk (RW) bridges and Schur process. I'll build from first principles. 
%Not talk Gibbsian line ensemble 
I'll talk about the structure as determinantal point process.
\item
Airy line ensemble, the universal object.
\item
KPZ line ensemble, and use it to understand temporal correlation of KPZ SPDE, a well-known stochastic PDE
\end{enumerate}•

%integrable origin, no time

\subsection{Non-touching geometric random walks}

Fix $M,N\ge 1$ and $a_1,\ldots, a_M;b_1,\ldots, b_N>0$, $a_ib_j<1$ (this comes from a normalization).

Hold the starting and ending point constant. 
Do a random walk, at times $1,\ldots, M+N-1$, have jumps with parameter $a_1,\ldots, a_M,b_N,\ldots, b_1$. Choose each of the jumps with a geometric distribution with the corresponding parameter.
%For each height increase, get power the parameter to the height change.
%Condition on ending at a certain height.

More precisely, recall $X\sim \mathsf{geo}(q)$ if $\Pj(X=k)=(1-q)q^k$, $k\ge 0$. Then the $\mathsf{geo}(\vec a;\vec b)$ RW bridge is 
\begin{align*}
Y(s)-Y(s+1)\sim \begin{cases}
\mathsf{geo}(a_s),&s=1,\ldots, M,\\
\mathsf{geo}(b_{N+M-s}),&s=M,\ldots, M+N-1
\end{cases}•
\end{align*}•
We consider $\{Y_i\}_{i=1}^\iy$ of $\mathsf{geo}(\vec a;\vec b)$ bridges conditioned on:
\begin{itemize}
\item
$Y_i(0)=-i=Y_i(M+N)$
\item
Non-touching: they need to keep distance 1 apart (no touching at a corner).
\end{itemize}•
There are finitely many excited curves (curves that are not flat), because the number of excited curves is $\le \min\{M,N\}$, and once a curve is flat, all subsequent curves are flat.

Picture when $M$ is fixed and $N\to \iy$: look at a sloped window of width $M^{2/3}$ and height $M^{1/3}$, and rescale. What happens when $M\to \iy$? As $M\to \iy$, this converges to the Airy line ensemble (which is a stationary object) minus a parabola. I'll explain this in the first 2 lectures.

%nice limit shape with slope
\begin{enumerate}
\item
Today we'll connect the bridge to a Schur process, and use this to relate to a determinantal point process, and get convergence for finite dimensional distributions. This proves convergence at any fixed number of times. This doesn't give functional convergence, tightness, invariance under resampling.
\item (tomorrow)
Tomorrow we'll show the Gibbs property, which results in tightness/functional convergence, which gives the Airy line ensemble.
%condition on noncrossing
\end{enumerate}•

\paragraph{Schur process} It's the connection to Schur process that allows us to do calculations.
Define $\la_i(0)\equiv 0$, and call $\la_i(s)=Y_i(s)+i$. For each $s$, $\la(s) = (\la_i(s))_i$ forms a partition:
\begin{align*}
\la_1(s)\ge \la_2(s) \ge\cdots \ge 0\text{ integers}
\end{align*}•
For example, $(4,4,2,1,1,0,0,0,\ldots)$ is a partition. 

The size of a partition $\la$ is $|\la|=\sum \la_i$. 

Two partitions $\la$ and $\mu$ are interlacing, $\la\ge \mu$ if $\la_1\ge \mu_1\ge \la_2\ge \mu_2\ge \cdots$

A \vocab{skew Schur polynomial} in one variable $a$ is
\begin{align*}
S_{\la/\mu}(a) &= \one_{\la\ge \mu} \cdot a^{|\la|-|\mu|}.
\end{align*}•
\begin{exr}
If $\{Y_i\}$ is a $\mathsf{geo}(\vec a;\vec b)$ bridge, then it pushes forward to a measure $\{\la_i\}$ with the following form
\begin{align*}
\Pj_{\vec a; \vec b}(\vec \la) &= Z(\vec a;\vec b)^{-1} \prodo iM S_{\la(s)/\la(s-1)} (a_s) \prodo sN S_{\la(M+N-s)/\la(M+N-s+1)}(b_s).
\end{align*}•
Here, $Z(\vec a, \vec b)$ is normalization.
\end{exr}
This is cumbersome but useful. We had our ensemble of curves. At each time, we form a partition by looking at the point process and doing this affine shift. The measure on this collection is given by the interlacing times $a,b$'s raised to the differences.
\begin{exr}
The marginal law of $\la(M)$, called the \vocab{Schur measure} is given by 
\begin{align*}\Pj_{\vec a, \vec b}(\la(M)) &= Z(\vec a; \vec b)^{-1} s_{\la(M)}(\vec a) s_{\la(M)}(\vec b)
\end{align*}
where the \vocab{Schur polynomial} is
\begin{align*}
S_\la(\vec a) &= \sum_{(0,0,\ldots)=\phi =\la(0)\le \cdots \le \la(M)}\prodo sM S_{\la(s)/\la(s-1)} (a_s).
\end{align*}•
Show that 
\begin{itemize}
\item
it is symmetric in the $a$ variables
(This is not obvious; it is a beautiful fact about invariance of the system.)
\item
(bialternant formula) $s_\la(\vec a) = \fc{\det(a_i^{\la_j+M-j})_{i,j=1}^M}{\det(a_i^{M-j})_{i,j=1}^M}$. (These are Vandermonde determinants.)
\item
(partition function) $Z(\vec a; \vec b)=\prod_{i,j} (1-a_ib_j)^{-1}$. 
Cauchy-Littlewood identity: $\sum_\la S_\la(\vec a)S_\la(\vec b) = Z(\vec a;\vec b)$. 

Idea: prove something locally, 1-variable skew Cauchy-Littlewood identity. If I look at the product of 2 and sum, $\sum S_{\la/\ka}(a)S_{\mu/\ka}(b) = (1-ab)\sum_\nu S_{\nu/\la}(b) S_{\nu/\mu}(a)$. I can confirm this by hand. Interate this $mn$ times, get out a factor of $(1-ab)$ each time. Eventually you sum everything. This is a nice trick that generalizes. %this is easy to prove
\end{itemize}•
\end{exr}
%This is exactly what I get by summing out.
If I look at a middle slice, the probability can be written as a product of Schur polynomials, which themselves admit formulas with determinants. Next I'll describe the connection with determinantal point processes.

The Schur process is a determinantal point process. I'll use this for asymptotics.

%why geo? Nice description in terms of Schur poly, dpp.
%Carlon-Macgregor formula
%through that can do asymptotics

%can define with arbitary jump rules
%scaling limits, in complete generality, if have second moments, converge to Airy line ensemble
%I don't think it's been worked out. nice to see that result
%geometric: non-crossing, write in terms of determinants.

%3
\paragraph{Determinantal point processes}
I'll show the Schur process is a DPP, to give asymptotics.
\begin{thm}
For $\la\sim \mathsf{Schur}(\vec a; \vec b)$, $\wt Y =\{\la_i-i+\rc 2\}_{i=1}^\iy$, a determinantal point process on $\Z+\rc 2$ with correlation kernel $K(i,j)$ where 
%later show can be written explicitly with contour integrals
\begin{align*}
\sum_{i,j\in \Z+\rc 2} K(i,j)v^i w^{-j} &= \fc{Z(\vec a;v)Z(\vec b;w^{-1})}{Z(\vec b;v^{-1})Z(\vec a;w)}\sum_{k=\rc 2,\fc32,\ldots} \pf wv^k.
\end{align*}
%point process on the line
%poisson
%defined by the intensity. constant or function, number of points given by integral.
%given by 2-variable function, correlation kernel.
This means that $\forall n$, $\{x_1,\ldots, x_n\}\in \Z+\rc 2$ distinct
\begin{align*}
\Pj(\{x_1,\ldots, x_n\}\in \wt Y) &= \det(K(x_i,x_j))_{i,j=1}^n.
\end{align*}•
\end{thm}
This will allow us to access asymptotic information about the system. The kernel is not getting more complicated as the system size grows.

\begin{exr}
\begin{align*}
\Pj(\la_1\ge s) & =\sumz \ell\iy \fc{(-1)^\ell}{\ell!} \sum_{x_1,\ldots, x_\ell>s}\det(K(x_i,x_j))_{i,j=1}^\ell.
\end{align*}•
%integrate beyond point. inclusion exclusion
\end{exr}
There are two things left to do:
\begin{itemize}
\item
Show where this formula comes from.
\item
Show where this formula goes (asymptotics).
\end{itemize}•

We show where this formula comes from. 
The following theorem shows a certain class of measures is determinantal.
%sound like rmt theor
%progenitor of measures like the GUE from random matrix theory.
\begin{df}
An $N$-point biorthogonal ensemble is a probability measure on $\{x_1,\ldots, x_N\}$ of the form
%N point correlation function
\begin{align*}
\Pj_N(\{x_1,\ldots, x_N\}) &= c_N \det(\phi_i(x_j))_{i,j=1}^N \det(\psi_i(x_j))_{i,j=1}^N
\end{align*}•
under the condition that the Gram matrix $G_{ij} = \sum_x \phi_i(x)\psi_j(x)$ is finite, so that there is a normalizing constant $c_N$.
\end{df}
\begin{thm}
$\Pj_N$ is DPP with 
\begin{align*}
K(x,y) &= \sumo{i,j}N \phi_i(x)[G_{ij}]^{-\top} \psi_j(y)
\end{align*}•
\end{thm}
\begin{exr}
Prove this (use the Cauchy Binet theorem). %Expand in terms of products of signs of minors.
\end{exr}
%trace of kernel
%density at location.
\begin{exr}
Apply to Schur measure (bialternate formula). Prove the $K(i,j)$ generating function formula. (Hint: use Cauchy determinant.)
\end{exr}

\paragraph{Asymptotics}

What are we after? Let's go back to nonintersecting geometric random walks, and simplify by setting $M=N$, $a_i=b_j=q$, $N\nearrow \iy$. (Things actually simplify when we keep the parameters. The more parameters, the fewer options. The theory of Schur polynomials is natural. Then we degenerate.) %This is a general concept.

There is a limit shape evolving, with two parts. One moving into open air, one invading the space beneath.

What are some interesting questions about the picture?
\begin{itemize}
\item
Limit shape.
%midpoint. low density 1, high density 0. how interpolate
Consider the midpoint line. How does the density interpolate from 1 to 0?
It turns out that the bottom transition happens at $\fc{-2q}{1+q}N$; the top happens at $\fc{2q}{1-q}N$; and we can derive a formula for how it transitions.
%easier to go up than down
\begin{align*}
\text{Density}(u) &= \fc{\arg(v_+)}{\pi}
\end{align*}•
where $v_{+/-}$ are complex conjugate roots of 
\begin{align*}
f'(v)&=0\\
f(v) &= \log (1-q/v) - \log(1-qv)-u\log v.
\end{align*}•
(I'll try to prove this.)
\item
Local limits % (I won't prove this.)
\begin{itemize}
\item
In the bulk, it has a limit, the discrete sine process.
\item
At the edge, it has a scaling limit to the Airy point process.
It is a progenitor for proving the finite dimensional distribution results we're after.
\end{itemize}•
\item
Linear statistics, large deviation principles...
\end{itemize}•
I'll try to explain where this comes from.

$K(x,x) = \Pj(x\in \wt Y)$ is the density.

Given a generating function, by using Cauchy's residue theorem, you can rewrite in terms of a double contour integral.
\begin{exr}
For $M=N$, $a_i=b_j=q$ for all $i,j$, letting $f(v) = \log(1-q/v) - \log(1-qv) - u\log v$.
\begin{align*}
K(uN,uN) &= \prc{2\pi i}^2\oint\oint \fc{\sqrt{vw}}{(v-w)vw}
\exp\{N[f(v)-f(w)]\}\,dv\,dw
\end{align*}•
The contour over $w$ is a circle surrounding $q$; the contour over $v$ is outside the $w$-circle, and not containing $q^{-1}$.
\end{exr}
Density asymptotic: solve $f'(v)=0$. 
%solve: distance 
($f(v)$ as small as possible, $f(w)$ as large as possible, then can localize contribution of kernel. If I can't, I can find the main contribution. Given at critical points. So solving this is natural.)
Solve discriminant$=0$. $u=\fc{2q}{1-q}, -\fc{2q}{1+q}$.
\begin{itemize}
\item
$u>\fc{2q}{1-q}$:  Draw $\Re[f(v)]$: 
There are 3 curves upon which is 0. See picture
%\searrow1

As long as I'm in this region, $\text{density}_N(u)\searrow 0$ exponentially fast.
\item
$\fc{2q}{1+q}<u<\fc{2q}{1-q}$, $v_{+/-}$ complex conjugate pairs.
2 contours where $=0$, with 2 points of intersection, $v_+,v_-$. %+ - +. - outside
%If $v$ contour goes into positive region, can be big. But may be oscillations. want to avoid worrying.
Deform $v$ contour so that it sits in the $-$ region. $w$ is in the positive area. As I deformed, I encounter residues. We had $\int \fc{\sqrt{vw}}{(v-w)vw}\exp(\cdots)$. Using the residue theorem, get 
\begin{align*}
\text{density} &\approx \int_{v_-}^{v_+} \rc w \,dw - \int \int \cdots.
\end{align*}•
By the same argument as before, $\int\int\to 0$ at speed $\rc{\sqrt N}$. The first part gives $\fc{\arg(v_+)}{\pi}$.
\item
$u<\fc{-2q}{1+q}$. Density is $1-$(something going to 0 exponentially fast).
\end{itemize}•
%Summarize: We looked at this ensemble, and saw we can rewrite measure by Schur . 
%admit DPP, because Schur poly are determinantal
%idea about how to do asymptotics
%not what I advertised
%I showed easier versions of that
%something we forgot. 
%at beginning, use nonintersection rw.
%nowhere did I use that
%induce gibbs . how to resample conditioning.

%today more rmt

%brownian gibbs property
%tomorrow: prob argument using gibbs property

%continuum limit, q\to \iy, rescale partitions
%det of exp, take limit of corr kernels
%geo -> exponential (continuous)
%or Poisson limit q to 0, nonintersection Poisson rw.
%fix n, m\to \iy. bridges become Brownian, nonintersecting brownian bridges.
%all accessible through same tech

%Laplace's method : complex integrals. where is real contribution?
%write in terms of $e^{}$.
%hard: can oscillate.
%I'd like to find where $f$ is large. $\int e^{Nf(z)}\,dz$. Problem: can spin around a circle, and cancel out.
%Critical point: where imaginary part is not rapidly changing, can control with real part.

%enforce non-crossing
%Lvn to general, I don't think it works.

%how to simulate: RSK correspondence. go from NxM matrix of geometric random variables, ij, geo(a_ib_j). easy to produce. exact combo mapping from that to ensemble of nonintersecting bridges, pushforward is schur process. rsk built as simple agloirithm. other ways as Markov dynamics, monotone coupling
%https://arxiv.org/abs/1907.10160

\subsection*{2020/6/9 Lecture 2}
\subsection{Airy line ensemble}

%last time: dpp
%line ensemble from perspective of gibbs property
%functional objects

%other ensembles with useful gibbs property
%apply to question on temporal correlations.

%condition nontouching, start off staggered by 1
%eventually terminate in flat paths. limit shape that arises
%N, 2N

Top live at $\fc{2q}{1-q}N$.
Bottom live at $\fc{-2q}{1+q}N$.
Study the edge of the ensemble, get Airy line ensemble. Take window of length $N^{2/3}$, (mesoscopic compared to the order $N$) and height $N^{1/3}$. Take a finite number of times, subtract off a parabola to center, showed convergence of the finite dimensional distribution to limiting DPP.

Call the limiting ensemble $\cL_1,\cL_2,\ldots$ (starting from the top curve). We don't know there exists a continuous distribution. Prove convergence of random walk, start with finite dimensional distributions, then strengthen to functional CLT. How to strengthen from FDD to functional CLT?

Let $\cal A_i(\cdot)=\cL_i(\cdot)+x^2$. The $\{\cal A_i\}$ are the Airy line ensemble. %(The original curves are the Airy line ensemble minus a parabola.)
There are constants. % depend on location. 
The reason you have parabolic behavior because the limiting shape has nonvanishing second derivative.

\begin{thm}
\begin{itemize}
\item
The above convergence can be strengthened to hold for curves, in locally uniform on compacts topology.
\item
The Airy line ensemble 
$\cal A_i=\cL_i(x)+x^2$ is stationary process in $x$, with DPP FDD's.
%collection of random curves
\item
$\cL$ enjoys non-intersecting Brownian Gibbs property:\footnote{This comes from conditioning random walks from non-intersection. The property translates into the limit.} %was exactly present in the 

Fix $k_1\le k_2$, $a<b$. Then 
\begin{align*}
&\text{Law}(\cL_{[k_1,k_2]}[a,b]|\si_{\text{ext}})\\
&= \text{Law}(\cL_{[k_1,k_2]}[a,b]|\si_{\text{boundary}})\\
&= \text{Law}(k_2-k_1+1\text{ Brownian bridges conditioned on non-int. with themselves and }\cL_{k_1-1}[a,b],\cL_{k_2-1}[a,b])
\end{align*}
Here, $\si_{\text{ext}}$ has all information external to $\cL_{[k_1,k_2]}[a,b]$, and the first inequality says the only dependence is on boundary data.
\end{itemize}•
\end{thm}
This is why it's called a Gibbs property. In lattice models, if you restrict to a region, the measure inside only depends on the boundary. There are only local interactions.

%Vertical variable is index of curve. Horizontal variable is time, continuous.

We'll sketch a ``proof" of this theorem from geometric RW and give applications.

\begin{exr}
A Brownian bridge (BB) enjoys a simple Brownian Gibbs property. The law between $a,b$ is the Brownian bridge connecting the endpoints.

Use this to prove a.s. uniqueness of maximizer of BB.
%stochastic analysis
\end{exr}

One of the reasons the Giibs inequality is so useful is that it gives rise to a comparison inequality. 
There are others, like a FKG inequality that holds for this model, but I'll talk about this particular inequality
%\begin{enumerate}
%\item
\subsubsection{Stochastic monotonicity}

%Measure of all pairs of B
If boundary data is coupled, then the same holds true for the law of nonintersecting Brownian bridge measure:
If one set of boundary data is $\ge$ another set, then we can couple the conditional measures on Brownian bridges to also be ordered.

A similar monotonicity holds for non-touching geo RW bridges.

%lowest datat that will connect
%run MC now
%init condition ordered
%MCMC, Metropolis dynamics to update configurations, in way that ... ordering.
Poisson clock(x,k): When it rings, I flip a coin and change the height of each curve by $\pm 1$ if it leads to non-touching.
%midpoint rings, +1
(Doing that for one curve might violate non-touching, then for that curve I don't make the move; but if another doesn't violate, then move that one.)
\begin{exr}
\begin{itemize}
\item
Show this preserves order.
\item
Invariant measure $=$ geometric RW bridge on touching.
%does it occur for other RW?
\footnote{For general random walks, the condition on existence of this coupling is log-convexity.}
%worked out in continuous context
\end{itemize}•
\end{exr}
\begin{exr}
Find a discrete RW (with 3 different choices) which violates stochastic monotonicity. In particular, for a 2-step bridge going from $0\to 0$ and $0\to 1$, there doesn't exist a coupling such that the bridge $0\to 0$ is $\le$ the bridge $0\to 1$.
\end{exr}
%\item 
%Construct the Airy line ensemble (4 steps)
%
%Step 1: FDD convergence (Schur process). 
%\end{enumerate}•

\subsubsection{Construct the Airy line ensemble (4 steps)}

\paragraph{Step 1: FDD convergence (Schur process). }

This doesn't rule out exceptional behavior at random times.

%spurious events occurring at random times

\paragraph{Step 2: No big max.} Show on $[a,b]$ the top curve cannot get too high.
%et indexed by N. of geo rand walk
%tightness.
%what happens in between doesn't blow up either.
%stoch monotonicity translates to lower curve

Consider a top curve between $a$ and $b$. Let's say that ``high" means getting to $R$. 
%I'll argue that they sit 
Define a stopping time $\chi$, the first time between $a$ and $\fc{a+b}2$ that I exceed $R$, 
\begin{align*}
\chi &= \inf_{x\in [0,\fc{a+b}2]}\set{x}{\cL_1^{(N)}\ge R}.
\end{align*}•
(By continuity, it equals $R$ at that point.) Say the surve is above $-R$.
The Gibbs property holds true for random times, the Strong Gibbs property. 
%knowledge of where it is is measurable with respect to the external field. 
$[\chi, b]$ is a stopping domain: knowledge of where it is is measurable with respect to the external field. 
The Strong Gibbs property says that on $[\chi,b]$, the law is that of the BB with $BB(\chi)=R$, $BB(b)=\cL_1^{(N)}(b)$ and on avoiding $\cL_2^{(N)}[\chi,b]$. 

If $\chi$ exists there is a big max. We show that it's unlikely that $\chi$ exists.

If we drop $\cL_2^{(N)}$, then the BB drops.
On the event $\chi$ existing, look at what happens in between, and compare it to the linear interpolation between $\chi$ and $b$. The slope is bounded below. 

%at midpoint, >R/2
A BB has probability $\rc2$ of exceeding its linear %midpoint 
interpolation at any given time. There is 50\% chance it will exceed linear interpolation $R/2$  at the midpoint $\fc{a+b}2$. On the event $\chi$ exists, it exceed $R/2$ with probability $\ge \rc 2$. Exactly distributed as original line ensemble. By tightness, probability is going to 0. So 
\begin{align*}
\Pj(\chi\text{ exist}) &\ge 2\Pj\pa{\cL_2\pf{a+b}2\ge \fc R2}.
\end{align*}•
The probability $\chi_1\pf{a+b}2\to 0$ as $R\to \iy$.

\begin{exr}
Use this argument to show that for BB with $B(0)=B(1)=0$, 
\begin{align*}
\Pj\pa{\max_{[0,\rc 2]} B(s)\ge R}&\le \Pj(B(1/2)\ge R/2).
\end{align*}•
\end{exr}
%rule out jump up high, control using behavior at midpoint.

%big max somewhere imply big max at deterministic location.
%search half of space , big there implies big at boundary.

We still need to establish tightness, not just that the max is not getting large. We need to argue curves are not getting too steep, because we want the distribution to be on continuous paths. 
Paths converge to something with absolutely continuous Radon-Nikodym derivative wrt Brownian motion.

\paragraph{Step 3: Good separation.}

If curves are not separated, Gibbs curve will sample small space, and there could be exceptional behavior.

%Suppose boundary conditions have good separation.

At two points, we have good separation.
This can be proved in two ways, using the Gibbs property, or DPP.

I want to argue I can take a little block with some width, and stick it between the bottom curve and point. 
Otherwise, the bottom curve becomes steeper, pushes itself up steeply, so we have issues (discontinuity) in limit. I need to rule this out.

%Forget these blocks.
Insert block at midpoint, spacer.

Shift picture over, and show it's true at $a,b$.

To show it's true at midpoint, use monotonicity argument.
%replace by... at fininte part of interval
Replace by a needle with sits at the midpoint, or the curve at the interval of the block.
The most the curves will drop is at most that conditioned on avoiding the interval. 

This gives the following type of problem. 
Take a Brownian bridge, add a spike in the middle, condition to go above spike. You want to show that the distance between the spike and the BB isn't going to 0. It's not 0 as long as the height of the spike is bounded by $R$.
%You need to argue 
There's an argument there. 

Then you have the blockers.
%logconcavity
%https://arxiv.org/pdf/1312.2600.pdf
%monotone coupling in S8.2 of KPZ ensemble, (143)

\paragraph{Step 4: Extract limit}
%RND
Show that the Radon-Nikodym derivative
\begin{align*}
\dd{\text{conditional measure}}{\text{free RWs}} &= \fc{\one_{\text{acceptance}}}{\Pj\pat{acceptance}}.
\end{align*}•
%normalize by prob of acceptance
Once I have good separation between paths, that implies tightness of the acceptance probability.
The argument is that once I have blockers, I can force my paths into places that don't intersect.
Force up into good region above $R$, and will avoid the bottom curve. So I have strictly positive probability of acceptance that will stay positive in the limit.

This implies tightness of the conditional measure. 
%tightness of RW measure.

Tightness of conditional measure and FDD implies functional convergence.

The Airy$-x^2$ has non-intersection BGP.

There's a lot of details.

\subsubsection{Application}
We went from discrete time regularity to regularity of the ensemble.
As an application:
\begin{thm}
The top curve %Airy 2-process
$\cL_1(x) = \cA_1(x)-x^2$ has a unique maximizer.
%johannsen
\end{thm}
The exercise was to show BB has the same property.  Once we know this, all we need to do is the following.
Take a big interval $[-R,R]$. The values there are small. 
%conditioned on second curve. stay above.
Top curve a.s. has unique maximizer. With probability $\to0$, there is no maximizer outside the interval. 
Show you're likely to be high $>-R^2/2$ in the middle, and likely to never go $>-R^2/2$ beyond $\pm R$. 
Take the union bound over intervals $[R,R+1]$, $[R+1,R+2]$,... 
Adapt the no big max approach. We need the probabilities to be absolutely summable, use tail bounds. $\cL_1(0)$ is distributed according to the Tracy-Widom GUE, 
\begin{align*}
\Pj(\cL_1(0)\ge s) &\le e^{-cs^{3/2}}\\
\Pj(\cL_1(0)\le s) &\le e^{-cs^3}.
\end{align*}•
We can also prove a localization result, $\Pj(|\text{maximizer}|>R) \le e^{-cR^3}$.
\begin{exr}
Fill in details.
\end{exr}
%discrete time continuous space

%blocks in good sep step

%I needed to argue that $\rc{\Pj\pat{acceptance}}$ is tight as $N\to \iy$. What does it mean for a probability to be tight? The probability is actually a random variable.
%Look at the conditional distribution that a BB starting at $a$ ending at $b$ will not touch the second curve.
%if big max, force up wouldn't be tight.
%separation rule out another issue. derivative  high around a, becomes steep for a while.
%what keeps the second curve doing such a crazy thing?
%devil's teeth.
%third curve also doing
%rule out acceptance prob low: most BB paths don't dodge like this.

%use that interval, can jump high in that interval
%as long as time $\eta$ to go up, 

%is there an easy modification of the non-touching RW model to make appear the other symmetry class GOE/GSE or even GbetaE?
%general beta, no Gibbs property. 
%Dyson beta, gauge transform, not true for other beta
%versions of ... related to
%best=1,2
%halfspace line ensembles
%0, paths from 0

\subsection*{2020/6/9 Lecture 3}

\subsection{KPZ line ensemble}
The law itself is a random variable, measurable with respect to everything external.

There are other line ensembles with other Gibbs properties.

Another continuum, line ensemble that doesn't have such a strict condition of ordering of paths.

I'll talk about something that has a soft Brownian Gibbs property (BGP), the KPZ line ensemble. 

Constructing infinite-volume Gibbs measures is not easy, it often takes a lot of work. You shouldn't think that if I specify any Gibbs property, there exists an infinite-volume version of it.
I won't give the construction, it goes though more sophisticated methods than the Schur process.

Then I'll use the KPZ line ensemble to give temporal correlation of the KPZ equation.

\subsubsection{KPZ line ensemble}

Unlike the Airy line ensemble,
%define in terms of finite dim dist
as a stochastic process, I can't define it as a function of iid noise. 
This admits a description in terms of spacetime white noise.

Let $\xi(t,x)$ be Gaussian spacetime white noise.
%little bit of stochastic PDEs.
%1d white noice. cov

An impressionistic definition of spacetime white noise: If I have 2 regions, the covariance of the integrals is the area of the overlap. $\si(t,x)$ is a generalized function with negative regularity. 

I'll define something related to directed polymers. For $k=0$, let $Z_0(t,x)\equiv \one$; for $k\ge 1$ consider $k$ nonintersecting Brownian bridges $b_1,\ldots, b_k$ from $(0,0)$ to $(t,x)$. (A little work is needed to define that.)
%A little work needed to define that. condition on nonintersecting epsilon apart, take limit.
%imagine as poisson  point process. weak limit of that
\begin{align*}
Z_k(t,x) :&= p(t,x)^k
%heat kernel to kth power
\E_{b_1,\ldots, b_k} \ba{:\exp: \bc{\sumo jk \int_0^t \xi(s,b_j)\,ds}}.
%integral of white noise along path
\end{align*}•
This is a continuum partition function. The noise is not a continuous function, so it's hard to integrate along a random path. Consider smoothing the noise: replace $\xi$ by the mollified $\xi_\ep = \xi*\de_\ep$. %$\ep$ wide
There is some regularization that makes this a martingale:
\begin{align*}
:\exp:\text{ means }
\exp\bc{\int_0^t\xi_\ep(\cdot) - \de_\ep*\de_\ep(0)t/2} \text{ as }\ep\searrow 0
\end{align*}•
I won't use this description, I'll use a fact about it which can be proved independently.
$Z=Z_1$ soles the SHE
\begin{align*}
\pl_t Z &= \rc2 \pl_{xx} Z+\xi Z
\end{align*}•
(HJ equation with additive white noise)
This is related to directed polymers, related to Branching/dying RWs in random environments, density evolution in random environment in algae or disease.
$h=\log Z_1$ solves the KPZ equation
\begin{align*}
\pl_t h &= \rc 2\pl_{xx}h + \rc2 (\pl_xh)^2 + \xi,
\end{align*}•
about stochastic interface grown.
For $k\ge 1$, $t$ fixed, define KPZ$^t$ line ensemble
\begin{align*}
h_k^{(t)} (x) &= \log\pf{Z_k(t,x)}{Z_{k-1}(t,x)}.
\end{align*}•
In particular, $h_1^{(t)}=h(t,x)$ is exactly the KPZ (``narrow wedge"). 

Look at the collection $\{h_k^{(t)}(\cdot)\}_{k=1}^\iy$. They are not necessarily ordered. They are parabolic because $h(x)+x^2$ %x^2/2?
is stationary in $x$. This is a cute way to see it: in the definition of $Z_K$, there is the heat kernel $p(t,x) = \rc{\sqrt{2\pi t}}e^{-x^2/(2t)}$. If I do an affine shift, I get back the law of $k$ BB's with different starting and ending points. White noise is also invariant to the shear. Stationarity is easy. The important point is that there is a hidden symmetry/invariant. 

Going to the enhanced structure---looking at the bigger structure, embedding in the bigger ensemble---I have a Gibbs property.
\begin{thm}
For $t$ fixed, KPZ$^t$ line ensemble has a soft BGP:
For any $k_1\le k_2$ and $a<b$ the law of $h_{[k_1,k_2]}^t[a,b]$ given $\si_{\text{ext}}$ only depends on the boundary data $h_{k_1-1}^t [a,b]$, $h_{k_2+1}^t[a,b]$, and is invariant under the following resampling procedure.

Choose $\cal U\sim \mathsf{Unif}[0,1]$ and sample $k_2-k_1+1$ non-intersecting Brownian bridges (non-intersecting with themselves) $b_{[k_1,k_2]}$ from $h_{[k_1,k_2]}^t (a)$ to $h_{[k_1,k_2]}^t(b)$. %look between each pair, compute interaction energy

Accept the first set of bridges for which 
\begin{align*}
\cal U&\le 
\exp\bc{
\sum_{k=k_1}^{k_2}\int_a^b H(h_{k-1}^t(s)-b_k(s)) + H(b_k(s)-h_{k+1}^t(s))\,ds
}%gibbs weight by exp
\end{align*}•
where $H(x)=H_1(x)$, $H_r(x)=e^{-rx}$. 
\end{thm}
Consider the case when $r=\iy$. 
Then the energy is 0 if x positive, $\iy$ if x negative.
Depending on the order, we get no energy, or infinite energy
If in order, no energy, if out of order, infinite energy.
If every go out of order, RHS is 0.
This recovers the nonintersecting property.
In the case of finite $r$ we get penalization in exponential form, integrated over time out of order.
Another way to interpret the formula is that it's a Radon-Nikodym derivative.

\begin{conj}
If we scale transversally $t^{2/3}$ and vertically $t^{1/3}$ we get a line ensemble which has the $H_{t^{1/3}}$ BGP. As $t\nearrow \iy$ this should converge to the Airy line ensemble minus parabola which has the $H_\iy$ BGP.
\end{conj}
%Top curve, 1 point distribution, converge to the 1 point distribution of the Airy line enseble. %TW GUE. 
%If we had uniqueness, using Gibbs property, we can extract out the KPZ.
%We don't know how to prove the uniqueness of the limit.

We know one point convergence of top curve.
\begin{thm}
...%\frac{h(t,0)+\fc{t}{24}}{t^{1/3}}
\end{thm}
\subsubsection{KPZ equation temporal correlation}

%How does KPZ growth decorrelate?

It models growth interfaces. There are bumps that propagate in time and die out. How does KPZ growth decorrelate in (a long) time? 
The KPZ equation is a universal model, but its universality is in its long-time behavior.

Belief: correlation decays in $\ep^{-3}t$ time scale, vary transversally $\ep^{-2}x$, scale of fluctuations is $\ep^{-1}$. The $2/3$, $1/3$ we saw earlier is here as well.

So we expect a nice scaling limit under this scaling:
\begin{df}
Define 3:2:1 scaled KPZ equation
\begin{align*}
h_\ep(t,x):&= \ep'\ba{h(\ep^{-3}t,\ep^{-2}t)+\fc{\ep^{-3}t}{24}}
\end{align*}•
\end{df}
We know one point GUE-TW convergence and (via $H_1$-BGP) spatial tightness. What about temporal behavior?

%spatial tightness
%here is some lichen doing its best to model the math problab.ca/louigi/temp/lichen.jpg

\begin{align*}
\Corr(X,Y) &= \fc{\E[(X-\ol X)(Y-\ol Y)]}{\sqrt{\Var(X)\Var(Y)}}
\end{align*}•
%correlation between height fluctuation at 1, 1+\be
%as \be\to 0, \to 1. as \be \to \iy, \to 0.
%could be weird if scaling wrong, miss the transition between correlated and uncorrelated.
%this shows that 3 is the right exponent.
\begin{thm}
For $\ep\searrow 0$, 
\begin{align*}
\Corr(h_\ep(1,0),h_\ep(1+\be)) &\approx
\begin{cases}
\Te(\be^{-1/3})&\text{as }\be\nearrow \iy\text{ (remote)}\\
\Te(\be^{2/3})&\text{as }\be\nearrow \iy\text{ (remote)}\\
\end{cases}•
\end{align*}•
and any subsequential limit of $h_\ep(1,x)$ is H\"older $\rc2-$ in space and $\rc 3-$ in time. 
\end{thm}
%holder cont in space 1/2
%1/3 in time
%different smoothness time in short scale and longscale
%different beast
(conjecture...)

\subsubsection{Idea of proof}

Three ingredients:
\begin{enumerate}
\item
Two-time variational formula. 

Show relation between 2 times. 
\item
Use KPZ$^t$ line ensemble to study.
\item
One point tail bounds.
\end{enumerate}•

\paragraph{(1) Variational formula:} For $s<t$ and $x,y\in \R$, let
\begin{align*}
Z(s,x;t,y) &- \E_{b(s)=x, b(t)=y} \ba{
:\exp: \bc{
\int_s^t \xi(t,b(r))\,dr
}
}
\end{align*}•
Then for every $s<t$, (Chapman-Kolmogorov property)
\begin{align*}
Z(t,y) &= Z(0,0;t,y) = \int_{\R} Z(0,0;s,x) Z(s,x;t,y) \,dx
\end{align*}•
%additive, split up. spacetime white noise indepdent
%i fno noise, just semigroup property
The two $Z$'s are independent. 
%Time $t$ to time $x$.
%reversible. 

Expressing everything in terms of $h$ we get
\begin{align*}
\text{Law}\{h_\ep(1,0), h_\ep(1+\be,0)\} &= \text{Law} \bc{
h_\ep(1,0),\sup_x {}_\ep(h_\ep(1,x), \wt h_\ep(\be,-x))
}
\end{align*}•
where $\wt h_\ep(\be,\cdot)\stackrel d=\be^{1/3} h_{\be^{1/3}\ep}(1,\be^{-2/3})$ (independent copy) and
\begin{align*}
\sup_x{}_\ep f(x) :&= \ep \log\pa{\int_\R e^{\ep^{-1}f(\ep^2x)}\,dx}\xra{\ep\searrow 0} \sup_x \be(x).
\end{align*}•
See pic.
%2 indep solutions, 1 scale in diffusive manner wrt 
%if \be large, 2nd indep curve getting wider.
%location of maximizer stays tight. It doesn't scale away with $\epsilon$.
%The maximizer. 
%value at 0 and x^* not too different. If I can do, then can control the correlation.
%joint distribution, value of 0 of bottom, and the sum at maximizer $x^*$. 
Let's imagine $\sup_\ep\to \sup$ and consider Airy $\cL$ instead on $h_\ep$. Focus on $\be\nearrow \iy$ case:
\begin{align*}
\Corr(\cL(0),\cL(x^*)+\be^{1/3}\wt (\be^{-2/3}x^*))
\end{align*}•
Everything boils down to controlling small-scale oscillations.
\subsubsection{Small scale oscillations}
Claim:
\begin{align*}
\Pj\pa{\sup_{x\in [0,\eta]} |\cL(0)-\cL(\eta)|\ge s\eta^{1/2}||\cL\{-2+\eta,0,\eta,2\}|\le s}
&\le ce^{-cs^2}.
\end{align*}•
%right scale for brownian motion
(Condition on 4 deterministic times. Pay a small amount to control. That I can do using tail bounds.)
Extreme jumps on small intervals has Gaussian-type decay.
Two parts:

\paragraph{Fall:} %Top curve of line ensemble. Value at 0, 2 controllable. 
%2nd curve prop up 1st curve.
%if remove, drop, at most to being brownian bridge.
%slope is at most s
%BB with slope s, on small 0 to eta interval, decrase by >s\eta^{1/2} is exponentially small.
%scaling os slope is $\eta^{1/2}$. For $s$ small, $s\ll \eta^{1/2}$. This shows prob of quick fall is gaussianly small.

\paragraph{Rise:}
%monotonicity doesn't help us. 2nd curve props up, may induce rise. No big max ruled out getting big at random time. Beneath can push you up. If happens, then something bad at deterministic time.
%bounded below by brownian bridge, 1/2 chance...
%stopping time unlikely
%use strong gibbs property on \chi,2. A change between 0 and \eta by $s\eta^{1/2}$ is still a big change, unlikely by reversed fall argument.


%prob tools and integrable tools
\subsubsection{Questions and directions}

\begin{itemize}
\item
Origins and other integrable Gibbsian line ensembles:

All stochastic vertex models and their degenerations can be embedded as top curve of discrete Gibbsian line ensembles. Consequence of dynamics which preserve spin HL/q-Whitter process
%whittaker?
%tool: even if don't know how to take asympts, can take limit
\item
Other initial data:
\begin{itemize}
\item
One point distribution expressible via variational formula
\item
Hammond's Patchwork Quilt
\item
Dauvergne-Orthman-Virag construct Airy sheet from Airy line ensemble
%Airy line ensemble is the central object
\end{itemize}•
Airy line ensemble contains the entire KPZ fixed point.
\item
Uniqueness:

Conjectured that Airy line ensemble is characterized by its Gibbs property, stationarity, and ergocidicity.
%shift mode out by
%nice, show that things go to ALS without needing to study formulas.

Dimitrov-Matetshi prove it is characterized by its top curve and the Gibbs property.
%if only prove top converge, can show for entire
\end{itemize}•

%https://web-eur.cvent.com/event/44bdce4a-9414-4c7f-928f-4ea3e4985835/summary?RefId=Fellows

%nonint RW/BM in presence of wall
%spatial invariance issue

%other line ensembles constructed with other kernels appearing in RMT (sine, Pearcey, Bessel), included in Gibbsian line ensembles?

%reps in DPP
%in terms of Gibbs properties
%how understand relationship between DPP/Gibb
%functionals of nonintersecting line ensembles

%rewrite DPP in terms of how act on curve on multiplicative functionals
%Gibbs properties...